\chapter{An Introduction to Loci}

Loci is a programming framework for computational simulation of
physical fields.  It is particularly well adapted for computational
fluid dynamics.  It provides a descriptive rather than imperative
programming style.  The programs emitted by Loci satisfy these
descriptions, as the locus of a geometrical description is a figure
satisfying that description.

Loci intends to help a developer manage the complexity of a software
project.  Loci appeals to logic programming, and to ideas about
database queries, to manage complexity.  The developer provides
descriptions of objects and algorithms.  The user of a code developed
in Loci sets the code going by calling the program with a particular
query.  Backward chaining, Loci creates a schedule of computations.

Loci offers parallelism to the scientist or engineer without requiring
her to become a programmer or a computer scientist.  Whether in
multiple threads on a shared-memory machine, or in multiple processes
on a distributed-memory system, Loci schedules manage parallel
computation.

The user of Loci writes a program with three parts:  the definitions,
the transformations, and the goals.  The definitions are collected in
the facts database, the transformations are collected in the rules
database, and the query, as already mentioned, is given when the
program is called.

Now let us begin to learn about Loci by looking at an example.  We
shall develop code to numerically solve a two-dimensional heat
equation with initial and boundary conditions.
