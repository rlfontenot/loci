\chapter{Using Third Party Libraries}

As is often the case when writing high-performance numerical simulation
software, there may come a time when it is desireable to utilize some third
party's existing set of library routines rather than having to write your
own versions.  For example, the Portable, Extensible Toolkit for Scientific
Computing (PETSc), from Argonne National Laboratory, is a highly optimized
library for the numerical solution of partial differential equations on
parallel and serial architectures.  The library includes a large suite of
tools for solving a wide variety of linear and nonlinear systems of
equations.

As described in the preceeding chapters, Loci expects first class objects
to have methods defined to accomplish data serialization as well as input
and output.  Since most third party libraries will make use of their own
set of data structures, Loci will not be able to effectively manage them
as it does for first class objects which it is already familiar with.  So,
in order to utilize third party libraries in the context of a Loci
application, the user has a choice:  to write all of the supporting code
that Loci expects, or to utilize the {\tt blackbox} container to hide the
implementation details of the third party library from Loci.

\section{Blackbox Usage}
The following example shows how the blackbox container can be utilized
in order to make use of a third party library (in this case PETSc) with a
minimum of extra effort.

\subsection*{main.cc}
\input{petsc1_main_cc}

\subsection*{rules.cc}
\input{petsc1_rules_cc}
