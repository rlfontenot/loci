
\chapter {Modules}

In order to improve code maintenance and prevent tight coupling of
multi-disciplinary simulations, Loci uses shared modules. The rules
which does the standard computations are separated into modules so that they
can be reused by other applications. The shared modules can be loaded
into a particular namespace (for example nspace) . Once loaded, all
the variable names in the rules have nspace\@ prepended to it. If a
hierarchy of namespaces are used then the string \@nspace1\@nspace2\@...
is prepended to the variable names. 
%
In the case of multi-disciplinary simulations, there could be some
variables that are common between the modules or some variables that
needs to be imported from a module and some variables that need to be
exported to other modules. The variables that need to be
exported/imported are to be determined by the application
developer. In order to specify this kind of information to Loci, the
developer needs to register a particular module. A typical register
rule for the module is given below. 

\begin{verbatim} 
  class reg_mod_heat : public Loci::register_module {
    public:
    std::string  using_nspace() const ;
    std::string input_vars() const ;
    std::string output_vars() const ;
    } ;

  // In this example we are loading in the rules from the metrics
  //module. If the metrics module in turn uses other modules, then
  //those modules are also loaded in recursively.  
  std::string reg_mod_heat::using_nspace() const { return
    std::string("metrics"); }  
  
  // These are the variables that need to be loaded in from other
  //namespaces

  std::string reg_mod_heat::input_vars() const { return std::string("newton_iter,stop_iter,newton_iter{n,it},newton_finished{n,it},UNIVERSE,EMPTY,OUTPUT{n,it},OUTPUT{n},timestep_finished{n},ncycle{n},stime{n},do_output{n}"); }
  
  // These are the variables that need to be exported to other
  //namespaces.

  std::string reg_mod_heat::output_vars() const{ return std::string("interface_temp,interface_hqf,interface_heat_flux"); } 
  
// create a register module object. 
reg_mod_heat rmh ;
}
\end{verbatim}

%
Right now the module management is incomplete. The way we deal with
import/export variables right now is by putting those in an unnamed
namespace.  

If you need to load a module into  another program, use the routine
{\tt load\_module} . {\tt load\_module} is an overloaded
function. There are two versions of the same call. If you need to load 
in all the rules associated with a module without dealing with
import/export variables then you use the following call. 

\begin{verbatim}
 void load_module(const std::string from_str, const std::string
 to_str, rule_db& rdb, std::set<std::string> &str_set) ;
\end{verbatim}

For modules which require an init function, we pass the fact database
as well as the init file name. 
\begin{verbatim}
  void load_module(const std::string from_str, const std::string
  to_str, const char* problem_name, fact_db &facts, rule_db& rdb,
  std::set<std::string> &str_set) ;
\end{verbatim}

The set of string {\tt str\_set} keeps track of the module names
loaded till then. This is used mainly in the case when there is a
hierarchical loading of modules.  

